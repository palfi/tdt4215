\chapter{Method}
\label{cha:method}

% This section describes how we chose to solve the assignment, and what methods
% we used.
%
% How we solved it
%
% - Preprocessing
%   - Legemiddelhåndboka
%     - parsed HTML
%       - removed all non-alphabetic characters
%     - created Java objects
%       - 3 sentences instead of 1 on headers
%     - save as json
%   - Icd-10 / Atc OWL file:

We first parse the OWL files using the OWL API. In order not to parse every time
we run the program, we save it as a serializable Java object.

The Java objects have a String field for each annotation of each class of the
OWL ontology.

%(code_formatted, code_compacted, icpc2_code, icpc2_label, rdf-schema#label,
%umls_tui, umls_conceptId, umls_atomId, umls_semanticType, exclusion, synonym,
%underterm, inclusion, rdf-schema#seeAlso)

Although we save all the annotations we just use label, underterm, synonym to
search for similarities between the patient cases and the ICD10 ontology.

%   - Patientcases
%     - nothing
% - ICD-10 codes

We use Lucene as a searching engine for matching similarities between the ICD-10
ontology and the patient cases. What we did was to create and index Lucene
documents, this documents has two main fields; a Lucene TextField with the
description of each class (label, underterm, synonym annotations) and a Lucene
StringField with the ICD-10 code. Lucene will only tokenize the TextField
content.

Then we use each sentence of the patient cases as a query, Lucene will retrieves
a list of all relevant documents ranked by score.

%   - Lucene
%   - Legemiddelhåndboka
%     - loading saved json
%     - adding the codes to Java objects
%     - save for later
%   - Patient cases
%     - each sentence
%     - not stored, doesn't take much time
% - MATChing patientcases to chapters
%   - create index of chapters’ ICD codes
%   - used the ICD codes of the patient cases as query string


% vim: set ts=2 sw=2 tw=80:
