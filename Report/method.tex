\chapter{Method} \label{cha:}


% This section describes how we chose to solve the assignment, and what methods
% we used.

How we solved it

\begin{itemize}
\item Preprocessing
	\begin{itemize}
	\item Legemiddelhåndboka
		\begin{itemize}
		\item parsed HTML
			\begin{itemize}
			\item removed all non-alphabetic characters
			\end{itemize}
		\item created java objects
			\begin{itemize}
			\item 3 sentences instead of 1 on headers
			\end{itemize}
		\item save as JSON
		\end{itemize}
	\item Icd-10 / Atc owl file
		\begin{itemize}
		\item parsed owl
		\item created java objects
			\begin{itemize}
			\item Each annotation is a field.
			\end{itemize}
		\item save as serialized java object
		\end{itemize}
	\item Patient cases
		\begin{itemize}
		\item nothing
		\end{itemize}
	\end{itemize}

\item ICD-10 codes
	\begin{itemize}
	\item Lucene
	\item Legemiddelhåndboka
		\begin{itemize}
		\item loading saved JSON
		\item adding the codes to java objects
		\item save for later
		\end{itemize}
	\item Patient cases
		\begin{itemize}
		\item each sentence
		\item not stored, doesn't take much time
		\end{itemize}
	\item Matching patient cases to chapters
		\begin{itemize}
		\item create index of chapters’ icd codes
		\item used the icd codes of the patient cases as query string
		\end{itemize}
	\end{itemize}
\end{itemize}

% vim: set ts=2 sw=2 tw=80:
