\chapter{Method}
\label{cha:method}


% --------------------------------------------------------------------------- %
% Preprocessing
% --------------------------------------------------------------------------- %
\section{Preprocessing}
\label{sec:preprocessing}

We started with parsing Legemiddelhåndboka. We parsed page at a time and created
a Chapter java object of each chapter. If a chapter contains sub-chapters they
will be saved in a list of chapter objects in the parents’ chapter object. We
made custom parser using the Jsoup library. When we had parsed and made java
objects of all the chapters we used stored these objects for later use using
JSON format. This is because parsing the Legemiddelhåndboka takes some time and
does not need to be done every time the system starts. We chose JSON format
because this is readable for humans which was handy during development.

We first parse the OWL files using the OWL API. The information we got from each
class in the OWL ontology we saved as serializable Java objects. The Java
objects have a String field for each annotation of each class of the OWL
ontology.

We decided not to preprocess any of the cases. This is because they were small
and easy to parse. We just copied the files from the docx file and created a
.txt file for each case.

For all the text we parsed we removed all non-alphabetic characters. This was
because we did not think those characters provides us with any valuable
information.

After this step we got all the necessary information from Legemiddelhåndboka chapters
and the text (from annotations) from the owl classes saved in java objects.


% --------------------------------------------------------------------------- %
% ICD-10 codes
% --------------------------------------------------------------------------- %
\section{ICD-10-codes}
\label{sec:}

We use Lucene as a search engine for matching similarities between the ICD-10
ontology and the patient cases. What we did was to create and index Lucene
documents, this documents has two main fields; a Lucene TextField with the
description of each class (label, underterm, synonym annotations) and a Lucene
StringField with the ICD-10 code. Lucene will only tokenize the TextField
content.

For each sentence in each chapter in Legemiddelhåndboka we used the index we
created to get the Lucene document(that contains the information from the the
owl class) that had the best score. From this document we retrieved the ICD-10
code. When we got all the ICD-10 codes for every sentence in each chapter we
updated the JSON file to also contain these codes, since these won’t ever change
as long as the text in Legemiddelhåndboka doesn’t change.

We found the ICD-10 code that matched each sentence in each of the patient cases
as well, but we did not bother storing these as this job is lightweight compared
to getting the ICD-10 codes for all the Legemiddelhåndboka chapters.

After this step we got a string of ICD-10 codes for every chapter and all the
patient cases.


% --------------------------------------------------------------------------- %
% Ranking
% --------------------------------------------------------------------------- %
\section{Ranking}
\label{sec:ranking}

Since we got ICD-10 codes for each chapter as well as ICD-10 codes for each
patient case we had to use these to get the most relevant chapter for each
patient case. The way we approached it was to create an Lucene index with all
the ICD-10 codes from each chapter. Then we query this index with the ICD-10
codes for each patient as a query string. The thought behind this is that if a
ICD-10 codes is relevant for many sentences in a Legemiddelhåndboka chapter as
well as relevant for many sentences in a patient case, then the chapter and the
case is most likely relevant to each other.


% vim: set ts=2 sw=2 tw=80
