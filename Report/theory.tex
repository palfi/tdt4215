\chapter{Theory}
\label{cha:theory}

% --------------------------------------------------------------------------- %
% Indexing
% --------------------------------------------------------------------------- %
\section{Indexing}
\label{sec:indexing}

Lucene work with Document objects, each Document have Fields, each Field is
analyzed differently, for instance: a TextField is for content we want tokenized
(the description part of the icd10 codes), and StringField is for content we
don't want tokenized (the icd10 codes).

There are two main parts in the indexing process:

\begin{itemize}
\item The Analyzer extracts tokens out of text to be indexed.

\item The IndexWriter is a key component in the indexing process, is the one
      that adds documents to the index.
\end{itemize}


% --------------------------------------------------------------------------- %
% Scoring
% --------------------------------------------------------------------------- %
% TODO typeset math equations
\section{Scoring}
\label{sec:scoring}

The documents are chosen by a Boolean Model and are scored by a Vector Space
Model (VSM).
% TODO abbr ref VSM
\[
	\operatorname{cosine-similarity}(q,d) = \frac{V(q) \cdot V(d)} {|V(q)|\ |V(d)|}
\]
%
Where \(V(q) \cdot V(d)\) is the dot product (intersection) of the weighted
vectors, and \(|V(q)|\) and \(|V(d)|\) are their Euclidean norms.\\
\\
For practical reasons the real formula that Lucene uses is:
%
\begin{equation*}
\begin{split}
	\operatorname{score}(q,d) =
	& \operatorname{coord}(q,d) \cdot
		\operatorname{queryNorm}(q)  \\
	& \cdot
		\sum_{t \operatorname{in} q} (
			\operatorname{tf}(t\operatorname{in}d) \cdot
			\operatorname{idf}(t)^{2} \cdot
			\operatorname{t.getBoost}() \cdot
			\operatorname{norm}(t,d)
		)
\end{split}
\end{equation*}
%
\begin{description}

\item[\(tf(t\operatorname{in}d)\)]
correlates to the term's frequency, defined as the number of times term \(t\)
appears in the currently scored document \(d\).

\item[\(\operatorname{idf}(t)\)]
stands for Inverse Document Frequency.
\[
	\operatorname{idf}(t) = 1 + \log(\frac{numDocs}{docFreq + 1})
\]

\item[\(\operatorname{coord}(q,d)\)]
is a score factor based on how many of the query terms are found in the
specified document.

\item[\(\operatorname{queryNorm}(q)\)]
is a normalizing factor, doesn’t affect to the ranking.

\item[\(\operatorname{t.getBoost}()\)]
is a search time boost of term t in the query q.

\item[\(\operatorname{norm}(t,d)\)]
encapsulates a few (indexing time) boost and length factors.

\end{description}

% vim: set ts=2 sw=2 tw=80:
