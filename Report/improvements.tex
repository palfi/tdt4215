\chapter{Potential improvements} \label{cha:potential-improvements}

This section describes possible improvements to the system. We mostly focus on
improvements on the system that we think would give better results, getting us
closer to the golden standard.

When we look at Legemiddelhåndboka we can see that there are specific paragraphs
in each chapter that provides the most accurate description of each of the
chapters. By not using all the text in the Legemiddelhåndboka when getting the
icd codes, and instead choose the paragraphs that are most accurate, we might
get better icd-10 codes for each chapter.

We could make a thesaurus which stores different versions of the same word or
words related to each other. One example is the words \emph{Sukkersyke} and
\emph{Sukkersyken}. These are basically the same words, but our system see them as
different words since when it searches it matches the whole word and not just
parts.

An improvement would to take a more semantic approach and use more of the
semantics in the icd-10 ontology. If we looked at which icd-10 classes that was
a parent of other classes the system might have gotten more precise.

Lucene gives the developer lots of opportunities to customize how it is ran. An
improvement could have been achieved by experimenting with different ways of
configuring Lucene, especially how it calculates similarities(scoring).

Even though we use an Norwegian analyzer with Lucene that is suppose to have
stop words we could have added our own stop words, which could make everything we
did with Lucene more accurate.

Another potential improvement would be to use more than one ontology for
classification. We could use a combination of atc and icd when matching
patient cases to chapters. If a patient case and a chapter in Legemiddelhåndboka
would match with the usage of both ontologies we would be more sure that this
was a good match than we would with just one ontology.

% vim: set ts=2 sw=2 tw=80:
